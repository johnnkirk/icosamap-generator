% file: icomap00.tex
\input vanilla.sty
\nopagenumbers
\pageheight{8 in}
\pagewidth{5.5 in}
\hcorrection{+.5 in}
\scaletype{\magstep1}
\scalefont \bf \magstep3 \andcallit \bighead
\centerline{\bighead Computer Map Production}
\bigskip\centerline{\bighead using an}
\bigskip\centerline{\bighead Icosahedral Projection Design}
\vskip 1.5in
\centerline{an exercise in}
\bigskip\centerline{\bf The Design Initiative}
\bigskip\centerline{prepared for}
\bigskip\centerline{\bf The World Game}
\vskip 2 in
\centerline{by}
\medskip\centerline{John Kirk}
\medskip\centerline{Los Angeles, January 1971}
\vfill\eject
\centerline{Acknowledgements}
\bigskip\bigskip\flushpar
I am deeply indebted to Ann Porch, writer and programmer-analyst, Los Angeles,
for her extensive contribution,
both to the physical production of the computer program and this paper,
and to my morale as I try to communicate what I have learned.
\bigskip\flushpar
I am grateful to Dr. Norman J. W. Thrower, Professor of Geography, UCLA,
for his interest in, and sponsorship of, this work.
\bigskip\flushpar
I wish to acknowledge the help of numerous members of the CCN staff
and the academic community at UCLA
upon whose time I presumed during the latter half of 1970
while coding and debugging the computer program.
\bigskip\flushpar
This work was accomplished with the aid of intramural funding
received through the auspices of Campus Computing Network, UCLA.
The development and debugging of the program
was done on the IBM 360 mod 91 computer at CCN.
\vfill\eject
\centerline{\bighead Table of Contents}
\bigskip\bigskip
\bigskip\flushpar 1. How this design relates to the problem .. 1
\bigskip\flushpar 2. How the computer works .. 4
\bigskip\flushpar 3. How this projection method works .. 7
\bigskip\flushpar 4. How this program works .. 9
\bigskip\flushpar 5. Before and after .. 11
\vfill\eject
\centerline{\bighead Computer Map Production}
\bigskip\centerline{using an}
\bigskip\centerline{\bf Icosahedral Projection Design}
\bigskip\bigskip\centerline{\bf Section 1}
\medskip\centerline{\bf How this design relates to the problem}
\bigskip\bigskip\flushpar
The design this paper describes, addresses itself
to the discrepancy between some existing needs that are
not being met and some existing resources that are not
being fully utilized.
\bigskip\flushpar
Most maps to which people have been exposed have
several considerable drawbacks.  Historically, these
drawbacks have been insignificant but more recently
only tolerable.  In the last fifty years, they have
become crucial bottlenecks to the flow of effective,
realistic thinking about worldwide events.
\bigskip\flushpar
One drawback is the distortion caused by the
uneven "stretching" and "contracting" of features in
one area of the globe relative to another.  This
distortion in the Mercator Projection, for instance,
has resulted in people having such ingrained experiences
as thinking Greenland to be as large as the United
States, or as thinking of Antartica to have a coastline
nearly as long as the Equator.
\bigskip\flushpar
Another form of distortion is the cutting of a
land mass into separate pieces in the process of
spreading the map out flat.  Those of us used to maps
published in the United States tend to see Asia in
our minds as way out on the fringes of the world,
part here, part there.
\bigskip\flushpar
A further drawback in the present day is the
East-West orientation of most traditional maps, with
North at the "top" of the world and South at the
"bottom" of the world.  This traditional presentation
of the map was certainly optimum for the hundreds of
years when the only world-wide traveling was done by
ship, using the East-West winds and ocean currents in the
temperate zones, and when indeed the only global thinking
being done was by admirals of the sea.
\bigskip\flushpar
In the U.S. today, even though it is becoming
ever more obvious that each individual's day-to-day
survival depends on the long-run survival and even
the prosperity of one hundred percent of mankind, we
still find it difficult to picture Asia, the focus of
the world's population, as being just over the horizon,
North of us, instead of mysteriously distant over land
and sea to the East or West.
\bigskip\flushpar
This paper introduces a particular implementation
of R. Buckminster Fuller's Dymaxion Sky-Ocean World
map drawn by computer.  The projection described has
the advantage of no noticeable "stretching" distortion,
and at the same time, no breaks in land masses.  The
particular orientation of the land masses in relation
to each other gives, in our age of air transportation,
a truer picture of the distances between points by
havint the North Pole and thus the shorter polar routes
directly measurable and at the center of attention.
\bigskip\flushpar
At present the high cost and lack of precision
involved in producing maps by traditional "hand-craft"
techniques results in even our most creative and
intelligent thinkers having grown up under the
crippling handicap of imprecise, localistic, planar
perceiving of their essentially spherical environment.
\bigskip\flushpar
Applying computer technology, through the icosahedral
projection, to cartography can add precision and
truer representation of geographical reality to traditional
static map production.  It can also provide man, for
the first time, with dynamic, flexible and interactive
image tools with which to perceive his world
as a whole.  Such tools might catalyse his powers of
intuition in conceiving of designs which could lead
him to physical success in Universe.
\vfill\eject
\centerline{\bf Section 2}
\medskip\centerline{\bf How the computer works}
\bigskip\bigskip\flushpar
Features on a globe or flat map are represented
graphically as shaded areas, lines and points.  Such
information can be represented in the computer even
though the computer does not store information directly
in these graphic forms.
\bigskip\flushpar
Computers are designed to store, retrieve and
transform numbers.  They do so with great speed and
effectiveness.  Fortunately, shaded areas can be made
up of lines; lines can be made up of points; and the
location of a point can be indicated by two numbers.
These numbers are called its coordinates, and are
measurements of its distance up or down from the middle
of the page and left or right from the middle of the
page.
\bigskip\flushpar
Change to a picture that you can define graphically
such as making the picture twice as big or moving it
three inches to the right, can also be represented
mathematically.  For instance, making the picture twice
as big can be represented in the computer by multiplying
by two the numbers representing the picture.  Moving
it three inches to the right can be represented by
adding three to certain of the numbers representing
the picture.
\bigskip\flushpar
At this point, a useful concept may be noted.  When
we look at small plastic airplanes or cars,  we are in
the habit of calling them "models" of things in the
real world.  Children learn about cars and planes by
playing with these models.  Designers of real cars and
airplanes use very precise models in their work.  We
relate to, manipulate, and otherwise study models in
order to better understand or deal with the real world
because the model is more manageable than is the real
world.
\bigskip\flushpar
In exactly a parallel manner, the numerical
representation previously discussed can function as
a model.  Such a "mathematical model" when placed in a
computer can function far more powerfully than other
kinds of models, since manipulating mathematical models
through the computer is more dynamic, flexible and
interactive.
\bigskip\flushpar
Since people relate rather ineffectively to
numbers, and relate spontaneously with great effectiveness
(intuition) to graphic images, in recent years
machines have been developed which can draw pictures
under computer control.
\bigskip\flushpar
At present there are two basic forms of peripheral
machines which produce graphic images.  One has a roll
of paper which can be moved back and forth lengthwise
over a rotating drum.  A pen which can be moved crossways
to the paper, and picked up or put down, draws the map
or picture.  The other graphic machine is like a TV
screen where an electron beam can be aimed at any point
on the screen and turned on whenever a visible mark is
desired.  A picture of the map is made up of many such
marks on the screen.  A permanent copy of the picture
can be kept by having the computer trigger a camera
each time it completes an image.
\bigskip\flushpar
These graphic machines are controlled very much
like any machine with which you are familiar.  The
computer feeds them numeric measurements for each
direction the pen, paper or electronic beam can move,
in the same way that you give your TV measurements of
loudness or brightness by twisting its knobs.
\bigskip\flushpar
People control the computer by typing in groups
of commands called "programs".  The computer is able
to respond to these commands in the same way that your
TV is able to respond to your knob-twiddling command,
"Speak louder!".  Knob-twiddling and button-pushing that
you are used to can be considered a language in the
same way that there are various computer "languages"
in which computer programs are written.
\vfill\eject
\centerline{\bf Section 3}
\bigskip\centerline{\bf How this projection method works}
\bigskip\bigskip\flushpar
There are two steps in the projection method used
to produce this map.  The first explodes the sphere onto
the icosahedron (a shell almost like a sphere, but made
of twenty faces, each an equilateral triangle), and
the second arranges the twenty faces of the icosahedron
into a flat map.
\bigskip\flushpar
Imagining some things might help to visualize
how this projection method works.  Imagine two globes,
one inside the other.  If they're lined up properly, you
can pass a thin rod through Los Angeles on the outer
globe, through Los Angeles on the inner globe, and it
will then go through the center of the globes.
\bigskip\flushpar
It's clear that the same will be true for any
other pair of matching points on the two globes.
\bigskip\flushpar
Now put two different sized icosohedra around a
globe.  You can easily see that if the two icosahedra
are lined up with each other, a rod from the center,
through any feature of the globe, will determine the same
point on both icosahedra.
\bigskip\flushpar
Step one of the projection works in this way,
locating each point on the icosahedron corresponding
to a point on the globe by passing a line from their
mutual center through the feature of the globe to find
the corresponding point on the icosahedron.  Note
that it does not matter what size globe or icosahedron
you use, or whether they are touching at any points.
\bigskip\flushpar
Once you have an icosahedron with the map drawn
on its faces, you can proceed with step two.
\bigskip\flushpar
Imagine taking each of the twenty triangles
separately, like tiles, and placing them in any
arrangement you like on a flat surface, making sure,
whenever you put two edges together, that the map
matches properly.  For the purpose of seeing ocean
currents or whale migrations, you might want an arrangement
that keeps the oceans as continuous or unbroken as
possible, and puts the land masses out at the fringes.
For studying populations of people on land, and their
movements and interactions, you might like to have
the land masses as unbroken as possible, with the paths
of interactions at the center.
\bigskip\flushpar
The arrangement presented here manages to keep
each land mass entirely unbroken, with the major world
transportation routes near the center, when the map
is fully put together.
\vfill\eject
\centerline{\bf Section Four}
\bigskip\centerline{\bf How this program works}
\bigskip\bigskip\flushpar
The projection program embodies the task of
reading information, point-by-point, in latitude
and longitude coordinates, and drawing the information
on paper according to the projection method described.
\bigskip\flushpar
The icosahedron can be placed in any orientation
with respect to the globe, by rotating it in any direction,
keeping its center the same as the globe's.  Therefore,
the program reads into the computer, before beginning,
the particular position desired.
\bigskip\flushpar
The two numbers, latitude and longitude, which
represent a point on the globe, are read.  They are
angles, or fractions of a circle's circumference, and
may be measured in degrees and fractions of degrees,
or in degrees, minutes, seconds and fractions of seconds,
or in radians.
\bigskip\flushpar
Next the program decides which triangle of the
icosahedron the point will fall in, and thus on which
surface of the icosahedron to project the point.
\bigskip\flushpar
Step one of the projection method previously
described is the applied.  It consists of two parts.
The first converts the latitude and longitude of the
point to rectangular (x1--y1) coordinates on the
icosahedral face where the vertical (y1) axis corresponds
to North-South on the globe.  The second applies a rotation
to get the point in coordinates (x2--y2) relative to a
pole of the icosahedron rather than that of the globe.
\bigskip\flushpar
Step two is applied next.  It consists of a
rotation (x3--y3) and then a translation (x4--y4) to
change the coordinates of the point from its position
relative to the center of its triangle, to its final
position in the finished map (x4--y4).
\bigskip\flushpar
The last thing the program does for each point is
to give the command to move the pen to the coordinates
just computed and supply information about whether or
not the connecting line from the previous point is to
be drawn.
\vfill\eject
\centerline{\bf Section Five}
\bigskip\centerline{\bf Before and after}
\bigskip\bigskip
{\narrower\it\flushpar
... There are a myriad of economic
trends and other ultimately vital or
lethal evolutionary events transpiring
today which are invisible to
all humanity only because they are
too fast or too slow for man to apprehend
and comprehend them.
\bigskip\flushpar
... Our computerized World Game is
designed to accelerate the too slow
and decelerate the too fast of all
the known vital trendings and thereby
to bring them dramatically within
popular consideration and our world
game's solution.
\bigskip\flushpar
... As the general system of vital
trends becomes visible and its components
are seen to integrate synergetically,
we also will begin to discern
ways of using the world's resources
to ever higher and more universal
human advantage.  We will soon
learn popularly how to play the game
to explore for ways in which we may
use the world's resources so that we
may be able to make our whole planet
successfully enjoyable by all humanity
without any human profiting at the
expense of another and without interfering
with one another, and how to
do so in the shortest possible time.
\bigskip\par
R. Buckminster Fuller
\smallskip\par
Denver, June 18, 1969\bigskip}
\bigskip\flushpar
The computerized mapping design described in
this paper accepts information about the location of
coastlines, cities, populations, resources, routes,
etc. in tabular form as input, and draws maps as output.
\bigskip\flushpar
Further work is presently underway to make this
process progressively more dynamic and interactive.
One immediate next step is to produce time-lapse
animated films of the map, showing speeded-up or
slowed-down trends comprehensive to the whole earth.
Another step is achieving the capability of viewing
the map on a TV screen, for instance, while shifting
back and forth among different "resource" or "need"
kinds of information on it for comparison.
\vfill\eject
\bye
